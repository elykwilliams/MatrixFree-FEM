\documentclass[svgnames]{beamer} % for xcolor https://latex.org/forum/viewtopic.php?t=2445 
\mode<presentation>
{
\usetheme{Warsaw}
\setbeamertemplate{page number in head/foot}[totalframenumber]
%\setbeamertemplate{footline}[frame number]
\setbeamertemplate{headline}{}
\setbeamercovered{transparent}
% or whatever (possibly just delete it)
}

\usepackage[english]{babel}
\usepackage[utf8]{inputenc}

\title{Matrix-free finite element method}
\author[K. Williams, A. Zhiliakov]{
	Kyle Williams \and
	Alexander Zhiliakov
	\vskip -1mm
}
\institute[UH] {
	\includegraphicsw[.3]{logo_uh.png}
}
\date[December 10, 2019]{COSC\,6365 Final Project, December 10, 2019}

% bold for everything
\usepackage{bm}
% lowercase mathcal font
\usepackage{dutchcal}
\hypersetup{
	colorlinks,
	allcolors=.,
	urlcolor=blue,
	filecolor=blue
}
% braces for subeqns and boxes
\usepackage{empheq}
% http://mirror.hmc.edu/ctan/macros/latex/contrib/mathtools/empheq.pdf
\newcommand*\widefbox[1]{\fbox{\hspace{1em}#1\hspace{1em}}}
% hl
\usepackage{soul}
\makeatletter
\let\HL\hl
\renewcommand\hl{%
	\let\set@color\beamerorig@set@color
	\let\reset@color\beamerorig@reset@color
	\HL}
\makeatother
% sub figures / grids of pictures
\usepackage{subcaption} 
\graphicspath{{./img/}} % includegraphics path
\newcommand{\includegraphicsw}[2][1.]{\includegraphics[width=#1\linewidth]{#2}}
\newcommand{\svginput}[1]{\input{img/#1}} % pdf_tex path
\newcommand{\svginputw}[2][1.]{\def\svgwidth{#1\linewidth}\input{./img/#2}} % pdf_tex path
% tables
\let\oldtabular\tabular
\renewcommand{\tabular}[1][1.5]{\def\arraystretch{#1}\oldtabular}
\usepackage{hhline}
\usepackage{multirow}
% \coloneqq
\usepackage{mathtools}
% math commands for convinience
\DeclareMathOperator{\argmin}{arg\,min}
% bold vectors
\newcommand{\vect}[1]{\boldsymbol{\mathbf{#1}}}

\newcommand{\bcell}{T}
\newcommand{\bmesh}{{\vect{\mathcal T}}}
\newcommand{\mmesh}{{\vect{\mathcal \tau}}}
\newcommand{\bfaces}[1][]{{\vect{\mathcal F}_{\text{#1}}}}
\newcommand{\mfaces}[1][]{{\vect{\mathcal f}_{\text{#1}}}}

%\newcommand{\LTwo}{{\mathbb L^2}}
%\newcommand{\lTwo}{{\mathcal l^2}}
%\newcommand{\HDiv}{{\mathbb H_\text{div}}}
%\newcommand{\Rn}[1]{{\mathbb R^{#1}}}
%\newcommand{\Pn}[1]{{\mathbb P^{#1}}}
%\newcommand{\LTwoSpace}[1][\Omega]{{\mathbb L^2\left({#1}\right)}}
%\newcommand{\HSpace}[1]{{\mathbb H^{#1}\left(\Omega\right)}}
%\newcommand{\lTwoSpace}[1][\Omega]{{\mathcal l^2\left({#1}\right)}}
%\newcommand{\HDivSpace}[1][\Omega]{{\mathbb H_\text{div}\left({#1}\right)}}
%\newcommand{\PnSpace}[2]{{\mathbb P^{#1}\left({#2}\right)}}

% precond
\newcommand{\Pbd}{\mathbcal P_{\text{BD}}}
\newcommand{\Pbt}{\mathbcal P_{\text{BT}}}
\newcommand{\Pmg}{\vect P_{\text{MG}}}
\newcommand{\Pmgv}{\vect P_{\text{MG(V)}}}
\newcommand{\Pilu}{\vect P_{\text{ILU(0)}}}

\newcommand{\USpace}{\mathbb{\vect U}}
\newcommand{\PSpace}{\mathbb P}
\usepackage{dutchcal} % lowercase mathcal font
\newcommand{\aForm}[2]{\mathbcal a(#1, #2)}
\newcommand{\bForm}[2]{\mathbcal b(#1, #2)}
\newcommand{\lForm}[1]{\mathbcal l(#1)}
\newcommand{\LSpace}{\mathbb L^2}
\newcommand{\HSpace}{\mathbb H^1}

% differentials
\newcommand*\diff{\mathop{}\!\mathrm{d}}
\newcommand*\Diff[1]{\mathop{}\!\mathrm{d^#1}}

\begin{document}

	\begin{frame}
		\titlepage
	\end{frame}

	\begin{frame}{Overview}
		\tableofcontents
	\end{frame}
	
	\section{Motivation}
	
	\begin{frame}{Stiffness matrix of Poisson problem}
		\begin{block}{FEM discretization}
			{\centering
				\begin{tabular}{ l c r }
					\begin{tabular}{c}
						$- \Delta u = f$ \\ 
						+ boundary conditions
					\end{tabular}
					& $\qquad \Longrightarrow \qquad$ & $\vect A \, \vect x = \vect b$ 
				\end{tabular}\par}
		\end{block}
		\vfill
		\alert{Goal}: solve the system in $O(n)$ operations!
		\vfill
		In 1D case w/ Dirichlet BCs, the stiffness matrix~$\vect A$ has a simple structure
		$$
		\frac{1}{h} \begin{pmatrix}
		2	& -1		&			&			& 0		\\
		-1	& 2			& -1		&			&		\\
		& \ddots	& \ddots	& \ddots	&		\\
		&			& -1		& 2			& -1	\\
		0	&			&			& -1		& 2       
		\end{pmatrix} \in \mathbb R^{(n-1) \times (n-1)}, \quad h \coloneqq \frac{1}{n} 
		$$
	\end{frame}

	\begin{frame}{Stiffness matrix spectrum}
		Eigensystem of the stiffness matrix~$\vect A$
		\begin{align*}
			\lambda_{\alert{i}} &= 4 n \sin^2 \left( \frac{\pi \alert{i}}{2 n} \right), \\
			\vect v_{\alert{i}} &= \left\langle \sin \left( \frac{\pi \alert{i}}{n} \right), \, \sin \left( \frac{2 \pi \alert{i}}{n} \right), \, \dots, \, \sin \left( \frac{(n - 1) \pi \alert{i}}{n} \right) \right\rangle^T
		\end{align*}
		can be divided into \textbf{oscillating} ($\alert i > \frac{n}{2}$) and \textbf{smooth} parts \\
		\begin{figure}
		\centering
		\only<1>{
			\includegraphics{n20}
			\caption{Spectrum, $n = 20$}
		}
		\only<2>{
			\includegraphics{n40}
			\caption{Spectrum, $n = 40$}
		}
		\only<3>{
			\includegraphics{n60}
			\caption{Spectrum, $n = 60$}
		}
		\end{figure}
	\end{frame}

	\begin{frame}{The problem with clustering}
		\textbf{Why simple iterations are not efficient}:
		\begin{itemize}
			\item Good at eliminating oscillating components of the error, yet terrible at smooth
			\item So you may need~$\gg n$ iterations to converge 
		\end{itemize}
		\textbf{Why Krylov solvers are not efficient}:
		\begin{itemize}
			\item CG:~$|| \vect e_k ||_{\vect A} \le 2 \left( \frac{1 - \sqrt{\kappa^{-1}(\vect A)}}{1 + \sqrt{\kappa^{-1}(\vect A)}} \right)^k || \vect e_0 ||_{\vect A} = 2 \rho^k || \vect e_0 ||_{\vect A}$
			\item $O(h^{-2}) = \kappa(\vect A) \to \infty$, so~$\rho \to 1$. This is a pessimistic bound, however, it is the case for bad-clustered spectrum
			\item Thus you will need $n$ iterations (and hence~$O(n^2)$ operations) to converge
		\end{itemize}
	\end{frame}

	\begin{frame}{The problem with clustering: CG}
		\begin{figure}
			\begin{subfigure}{.5\linewidth}
				\centering
				\svginputw[.8]{cg.pdf_tex}
			\end{subfigure}%
			\begin{subfigure}{.5\linewidth}
				CG error norm history, $n = 500$. \textcolor{Blue}{Blue}: ``clustered'' Poisson matrix; \textcolor{Orange}{Orange}: uniform spectrum with cond number as for Poisson; \textcolor{Green}{Green}: uniform spectrum w/ two outliers, $\lambda_1$ and $\lambda_{500}$, s.t. the cond number is 100 times bigger than for Poisson
			\end{subfigure}
		\end{figure}
		Note that (\textcolor{Green}{Green}) after several iterations CG starts to behave like extreme eigenvalues are not present~[1]
	\end{frame}
	
	\section{Diffusion-reaction problem}
	
	\begin{frame}{Model problem}
		We consider diffusion-reaction problem
		\begin{empheq}[left=\empheqlbrace]{align*}
			-\epsilon \, \Delta u + u						&= f,		& \vect x & \in \Omega,		& \\
			u												&= g_D,		& \vect x & \in \Gamma_D,	& \\		
			\epsilon \, \nabla u \cdot \hat{\vect n} + u	&= g_R,		& \vect x & \in \Gamma_R	&	
		\end{empheq}
		\begin{figure}
			\centering
			\includegraphicsw[.4]{divgrad_u.png}
			\caption{Exact solution}
		\end{figure}
	\end{frame}

	\begin{frame}{Linear solvers comparison (1\,/\,2)}
		\begin{table}
			\centering
			\caption{
				Iterations / time ($\|\vect r_i\|/\|\vect r_0\| < 10^{-12}$) as one refines the mesh, $\Omega_1 \subset \Omega_2 \subset \dots \subset \Omega_8$
			} 
			\label{tab:divgrad_iters}
			\footnotesize
			\begin{tabular}[1.2]{ c c c | c | c | c | c | }
				& & \multicolumn{5}{c}{\textbf{solver type}} \\ 
				\cline{3-7}
				& & \multicolumn{1}{|c|}{MG(V)} & MG(W) & $\Pmgv$CG & $\Pilu$CG & CG \\
				\cline{2-7} 
				\multirow{8}{*}{\rotatebox[origin=c]{90}{\textbf{mesh level}}} 
				& \multicolumn{1}{|c|}{$\Omega_1$} & 8\,/\,.025\,s  & 8\,/\,.018\,s  & 6\,/\,.024\,s & 16\,/\,.032\,s & 37\,/\,.007\,s   \\
				\cline{2-7}
				& \multicolumn{1}{|c|}{$\Omega_2$} & 9\,/\,.029\,s  & 9\,/\,.034\,s  & 7\,/\,.032\,s & 28\,/\,.033\,s & 72\,/\,.024\,s   \\
				\cline{2-7}
				& \multicolumn{1}{|c|}{$\Omega_3$} & 10\,/\,.035\,s & 9\,/\,.050\,s  & 8\,/\,.045\,s & 53\,/\,.039\,s & 139\,/\,.031\,s  \\
				\cline{2-7}
				& \multicolumn{1}{|c|}{$\Omega_4$} & 10\,/\,.079\,s & 10\,/\,.111\,s & 8\,/\,.077\,s & 103\,/\,.064\,s & 276\,/\,.140\,s  \\
				\cline{2-7}
				& \multicolumn{1}{|c|}{$\Omega_5$} & 11\,/\,.222\,s & 10\,/\,.385\,s & 8\,/\,.244\,s & 206\,/\,.267\,s & 546\,/\,.859\,s  \\
				\cline{2-7}
				& \multicolumn{1}{|c|}{$\Omega_6$} & 11\,/\,.864\,s & 10\,/\,1.09\,s & 8\,/\,.826\,s & 399\,/\,4.49\,s & 1036\,/\,6.30\,s \\
				\cline{2-7}
				& \multicolumn{1}{|c|}{$\Omega_7$} & 11\,/\,3.90\,s & 11\,/\,5.33\,s & 8\,/\,2.86\,s & 776\,/\,47.5\,s & 2033\,/\,73.9\,s \\
				\cline{2-7}
				& \multicolumn{1}{|c|}{$\Omega_8$} & 11\,/\,16.1\,s & 11\,/\,20.0\,s & 8\,/\,12.9\,s & 1498\,/\,360\,s & 3915\,/\,605\,s \\
				\cline{2-7}
			\end{tabular}
		\end{table}
	\end{frame}

	\begin{frame}{V-cycle visualization}
		\begin{figure}[t!]\tiny
			\centering
			\begin{subfigure}{.24\linewidth}
				\includegraphicsw{x_bar.pdf}
			\end{subfigure}
			\par\bigskip
			\begin{subfigure}{.24\linewidth}
				\includegraphicsw{x_initial.pdf}
				\caption{$P_h\,\vect x_{\text{initial}}$}
				\label{fig:divgrad_vcycle:a}
			\end{subfigure}
			\hfill
			\begin{subfigure}{.24\linewidth}
				\includegraphicsw{x_ssor1.pdf}
				\caption{$P_h\,\vect x_{\text{presmooth}}$}
				\label{fig:divgrad_vcycle:b}
			\end{subfigure}
			\hfill
			\begin{subfigure}{.48\linewidth}
				\begin{subfigure}{.48\linewidth}
					\includegraphicsw{x_corrected.pdf}
					\caption{$P_h\,\vect x_{\text{corrected}}$}
					\label{fig:divgrad_vcycle:c}
				\end{subfigure}
				\hfill
				\begin{subfigure}{.48\linewidth}
					\includegraphicsw{x_postsmoothed.pdf}
					\caption{$P_h\,\vect x_{\text{postsmooth}}$}
					\label{fig:divgrad_vcycle:d}
				\end{subfigure}
				\par\bigskip
				\begin{subfigure}{.48\linewidth}
					\includegraphicsw{x_ssor2.pdf}
					\caption{$P_h\,\vect x_{\text{SSOR\,2}}$}
					\label{fig:divgrad_vcycle:e}
				\end{subfigure}
				\hfill
				\begin{subfigure}{.48\linewidth}
					\includegraphicsw{x_ssor3.pdf}
					\caption{$P_h\,\vect x_{\text{SSOR\,3}}$}
					\label{fig:divgrad_vcycle:f}
				\end{subfigure}
			\end{subfigure}
		\end{figure}
	\end{frame}

	\begin{frame}{Linear solvers comparison (2\,/\,2)}
		\begin{figure}
			\centering
			\includegraphicsw[.8]{divgrad_time}
			\caption{}
		\end{figure}
	\end{frame}

	\begin{frame}{Convergence for different finite elements}
		\begin{table}\centering
			\caption{
				Linear~(\subref{tab:divgrad_conv:a}), quadratic Lagrange~(\subref{tab:divgrad_conv:b}), and Crouzeix\,--\,Raviart FE~(\subref{tab:divgrad_conv:c}); $i \coloneqq$ mesh level, $e_h \coloneqq \sqrt{\int_{\Omega_h} (u - u_h)^2 \diff{\vect x}}$
			} 
			\label{tab:divgrad_conv}
			\tiny
			\begin{subtable}[b]{.35\linewidth}
				\centering
				\begin{tabular}[.7]{ | c | c | c | }
					\hline
					$i$ & $e_i$ & $e_{i-1} / e_i$ \\
					\hline\hline
					0	& $7.12 \times 10^{-2}$ & --- \\
					\hline
					1	& $2.08 \times 10^{-2}$ & 3.43 \\
					\hline
					2	& $5.55 \times 10^{-3}$ & 3.74 \\
					\hline
					3	& $1.42 \times 10^{-3}$ & 3.91 \\
					\hline
				\end{tabular}
				\caption{}
				\label{tab:divgrad_conv:a}
			\end{subtable}%
			\begin{subtable}[b]{.3\linewidth}
				\centering
				\begin{tabular}[.7]{ | c | c | }
					\hline
					$e_i$ & $e_{i-1} / e_i$ \\
					\hline\hline
					$8.19 \times 10^{-3}$ & --- \\
					\hline
					$1.08 \times 10^{-3}$ & 7.58 \\
					\hline
					$1.38 \times 10^{-4}$ & 7.85 \\
					\hline
					$1.74 \times 10^{-5}$ & 7.92 \\
					\hline
				\end{tabular}
				\caption{}
				\label{tab:divgrad_conv:b}
			\end{subtable}%
			\begin{subtable}[b]{.3\linewidth}
				\centering
				\begin{tabular}[.7]{ | c | c | }
					\hline
					$e_i$ & $e_{i-1} / e_i$ \\
					\hline\hline
					$5.64 \times 10^{-2}$ & --- \\
					\hline
					$1.54 \times 10^{-2}$ & 3.66 \\
					\hline
					$3.95 \times 10^{-3}$ & 3.90 \\
					\hline
					$9.94 \times 10^{-4}$ & 3.97 \\
					\hline
				\end{tabular}
				\caption{}
				\label{tab:divgrad_conv:c}
			\end{subtable}%
		\end{table} 
		\begin{figure}
			\centering
			\begin{subfigure}{.35\linewidth}\centering
				\includegraphicsw[.8]{divgrad_l1_u.pdf}
			\end{subfigure}%
			\hfill
			\begin{subfigure}{.3\linewidth}\centering
				\includegraphicsw[.8]{divgrad_l2_u.pdf}
			\end{subfigure}%
			\hfill
			\begin{subfigure}{.3\linewidth}\centering
				\includegraphicsw[.8]{divgrad_cr_u.pdf}
			\end{subfigure}
			\caption{Corresponding FE solutions~$u_h$}
			\label{fig:divgrad_solns}
		\end{figure}
	\end{frame}
	
	\section{Oseen problem}
	
	\begin{frame}{Oseen problem}
		\begin{enumerate}
			\item After time discretization \& linearization of Navier\,--\,Stokes eqns, one gets
				\begin{empheq}[left = \empheqlbrace]{align*}
					\alert{\alpha} \, \vect u + (\alert{\vect w} \cdot \nabla) \vect u - \text{Re}^{-1}\,\Delta \vect u + \nabla p &= \alert{\vect f}, \\
					\nabla \cdot \vect u &= 0 \quad \text{\alert{(+ BCs)}}
				\end{empheq}
			\only<1>{
			\item Mixed weak form is
				\begin{empheq}[box=\widefbox]{equation*}
					\begin{split}
					&\text{find trial functions $\langle \vect u, p \rangle\in \USpace \times \PSpace$ such that} \\
					&\left\{
					\begin{aligned} 
					\aForm{\vect u}{\vect v} + \bForm{\vect v}{p} &= \lForm{\vect v}, \\
					\bForm{\vect u}{q} &= 0
					\end{aligned} 
					\right. \\ 
					&\text{holds for all test functions $\langle \vect v, q \rangle\in \USpace \times \PSpace$}
					\end{split}
				\end{empheq}}%
			\only<2>{
			\item Going to discrete spaces $\USpace_h$, $\PSpace_h$ leads to
				\begin{equation*}
				\adjustlimits 
				\inf_{q_h \in \PSpace_h} \sup_{\vect v_h \in \USpace_h} \frac{
					\bForm{\vect v_h}{q_h}
				}{
					\|q_h\|_{\LSpace} \|\vect v_h\|_{\left[\HSpace\right]^2}
				} \geq \beta > 0;
				\quad
				\underbrace{
					\begin{pmatrix}
					\vect A & \vect B^T \\
					\vect B & O           
					\end{pmatrix}
				}_{\mathbcal A \coloneqq}
				\begin{pmatrix}
				\vect\xi \\
				\vect\eta         
				\end{pmatrix}
				=
				\begin{pmatrix}
				\vect l \\
				O        
				\end{pmatrix} % \eqqcolon \vect b
				\end{equation*}
		}
		\end{enumerate}
	\end{frame}

	\begin{frame}{Physics-based preconditioning}
		Consider the block decomposition
		\begin{equation*}
		\mathbcal A = \mathbcal{L} \, \mathbcal{D} \, \mathbcal{U} = \begin{pmatrix}
		\vect I & O \\
		\vect B \, \vect A^{-1} & \vect I
		\end{pmatrix}
		\begin{pmatrix}
		\vect A & O \\
		O & \vect S
		\end{pmatrix}
		\begin{pmatrix}
		\vect I & \vect A^{-1} \, \vect B^{T} \\
		O & \vect I
		\end{pmatrix},
		\end{equation*}
		with $\vect S \coloneqq -\vect B \, \vect A \, \vect B^T$ being a pressure Schur complement. 
		\vfill
		Preconditioners:
		\vfill
		\begin{columns}
			\begin{column}{.5\textwidth}
				\begin{block}{Block-diagonal}
					\begin{equation*}
					\Pbd \coloneqq \widetilde{\mathbcal D^{-1}} = \widetilde{\begin{pmatrix}
						\mathbf A & O \\
						O & \mathbf S
						\end{pmatrix}^{-1}}
					\end{equation*}
					\vspace{5pt}
				\end{block}
			\end{column}%
			\begin{column}{.5\textwidth}
				\begin{block}{Block-triangular}
					\begin{equation*}
					\Pbt \coloneqq \widetilde{(\mathbcal D \, \mathbcal U)^{-1}} = \widetilde{\begin{pmatrix}
						\vect A & \vect B^T \\
						O & \vect S
						\end{pmatrix}^{-1}}
					\end{equation*}
					\vspace{5pt}
				\end{block}
			\end{column}
		\end{columns}
		\end{frame}
	
		\begin{frame}{Numerical results}
			\begin{table}\centering
				\caption{
					$\Pbd$BiCGStab and $\Pbt$BiCGStab for haemodynamics Stokes problem ($||\vect r_i|| < 10^{-10}$) for Taylor\,--\,Hood FE; numb of d.o.f. for  $\Omega_0$ is 556, and for $\Omega_5$ is 459\,635
				} 
				\label{tab:chd_stokes_iters}
				\begin{tabular}[1.2]{ c c c | c |}
					& & \multicolumn{2}{c}{\textbf{solver type}} \\ 
					\cline{3-4}
					\multirow{6}{*}{\rotatebox[origin=c]{90}{\textbf{mesh level}}} 
					& & \multicolumn{1}{|c|}{$\Pbd$BiCGStab} & $\Pbt$BiCGStab \\
					\cline{2-4} 
					& \multicolumn{1}{|c|}{$\Omega_1$} & 105\,/\,5.05\,s	& 38\,/\,2.71\,s \\
					\cline{2-4}
					& \multicolumn{1}{|c|}{$\Omega_2$} & 129\,/\,16.9\,s	& 41\,/\,5.35\,s \\
					\cline{2-4}
					& \multicolumn{1}{|c|}{$\Omega_3$} & 113\,/\,56.4\,s	& 43\,/\,20.4\,s \\
					\cline{2-4}
					& \multicolumn{1}{|c|}{$\Omega_4$} & 108\,/\,238\,s		& 43\,/\,84.9\,s \\
					\cline{2-4}
					& \multicolumn{1}{|c|}{$\Omega_5$} & ---				& 44\,/\,433\,s \\
					\cline{2-4}
				\end{tabular}
			\end{table}
		\end{frame}
	
		\begin{frame}{Solution postprocessing (1\,/\,2)}
			\begin{figure}
				\begin{subfigure}{.9\linewidth}
					\begin{subfigure}{1.\linewidth}\centering
						\includegraphics[height=70bp]{chd_u_0.png}\quad
						\includegraphics[height=70bp]{chd_u_0_zoom.png}
					\end{subfigure}%
					\par\bigskip
					\begin{subfigure}{1.\linewidth}\centering
						\includegraphics[height=70bp]{chd_u_1.png}\quad
						\includegraphics[height=70bp]{chd_u_1_zoom.png}
					\end{subfigure}%
				\end{subfigure}%
				\begin{subfigure}{.1\linewidth}\centering
					\includegraphics[height=100bp]{chd_u_bar.png}
				\end{subfigure}%
				\caption{
					Velocity fields for Stokes ($t = 0$) and Oseen ($t = 0 + \Delta t = .01$) problems
				}
				\label{fig:chd_u}
			\end{figure}
		\end{frame}
		
		\begin{frame}{Solution postprocessing (2\,/\,2)}
			\begin{figure}
			\begin{subfigure}{1.\linewidth}\centering
				\includegraphicsw[.7]{chd_p_0.png}
			\end{subfigure}%
			\par\bigskip
			\begin{subfigure}{1.\linewidth}\centering
				\includegraphicsw[.7]{chd_p_1.png}
			\end{subfigure}
			\caption{Pressure distributions for Stokes ($t = 0$) and Oseen ($t = 0 + \Delta t = .01$) problems}
			\label{fig:chd_p}
			\end{figure}
		\end{frame}
	
	\begin{frame}{Useful references}
	\begin{enumerate}
		\item \textbf{General theory}: M.\,Olshanskii \& E.\,Tyrtyshnikov ``Iterative methods for linear systems'' 
		\item \textbf{Numerical examples with geometric MG}: K.\,Mardal et. al. ``Software tools for multigrid methods''
		\item \textbf{Physics-based (or block) preconditioners}: \href{https://www.math.colostate.edu/~bangerth/videos.676.38.html}{Woflgang Bangerth's lecture} 
	\end{enumerate}
	\end{frame}

\end{document}


